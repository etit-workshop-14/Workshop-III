\begin{thebibliography}{99}
\bibitem{diode} \url{https://www.elektronik-kompendium.de/sites/bau/0201111.htm}, Abrufdatum: 08. Mai 2019.
\bibitem{ntc} \url{http://www.vishay.com/docs/29049/ntcle100.pdf}, Abrufdatum: 08. Mai 2019.
\bibitem{sensorik} Ekbert Hering; Gert Schönfelder. Sensoren in Wissenschaft und Technik. Springer (Berlin, Heidelberg), 2. Aufl. (2018)
\bibitem{optische} Martin Löffler-Mang. Optische Sensorik. Springer-Verlag (Berlin Heidelberg New York), 2012. Aufl. (2011)
\bibitem{schaltung_licht} \url{https://www.electronicsplanet.ch/Schaltun/lichtsc1/lichtsc1.html}, Abrufdatum: 20. Mai 2019.
\bibitem{funktionsweise} \url{https://www.baumer.com/ch/de/service-support/know-how/funktionsweise/funktionsweise-und-technologie-von-lichtschranken-und-lichttastern/a/know-how_function_lichtschranken-lichttasterl}, Abrufdatum: 20. Mai 2019.
\bibitem{elektro_masch} Fischer, Rolf; Linse, Hermann. Elektrotechnik für Maschinenbauer. Vieweg u. Teubner (Wiesbaden), 14., überarb. u. akt. Aufl. 2012
\bibitem{formelsammlung} Bernstein, Herbert. Formelsammlung. Springer Vieweg (Wiesbaden), 2., aktualisierte Auflage (2019)
\bibitem{halbleiter} Tietze, Ulrich; Schenk, Christoph; Gamm, Eberhard. Halbleiter-Schaltungstechnik. Springer (Berlin, Heidelberg), 12. Aufl. (2019)
\end{thebibliography}