\begin{abstract}
\section*{Abstract}
Ein sehr wichtiger Aspekt jeder Lehre ist die sowohl theoretische als auch praktische Beschäftigung
mit dem zu lernenden Stoff. Diesen Zweck erfüllten die Aufgaben dieses Workshops.
Wir hatten die Gelegenheit dazu, uns mit Sensorik in Form einer Lichtschranke und einer
Temperaturmessung in der Praxis auseinanderzusetzen und Messungen durchzuführen. Dabei
haben wir im Zuge der Recherche gelernt, welche Arten von Temperatursensoren es gibt und deren
Vor- und Nachteile ermittelt. Zudem haben wir Informationen zum Thema Arbeitsweise und Aufbau
typischer Lichtschranken gesammelt.
Im Zuge des Workshops wurden von uns verschiedene Messungen durchgeführt, Aufgaben gelöst
und Diskussionen durchgeführt, was unsere Zusammenarbeit förderte und das angeeignete Wissen
vertiefte. Mithilfe eines Mikrocontrollers und der entwickelten Schaltungen führten wir Messungen
durch, analysierten anschließend unsere Messergebnisse und formulierten daraus Resultate. Das
Lösen der Aufgaben hat des Weiteren unser logisches Denken gefordert.
\end{abstract}